\title{}
\subtitle{Foundational concepts}
\author{
	\begin{itemize}
		\item Counterfactuals
		\item Potential outcomes and consistency
		\item Causal identification assumptions
	\end{itemize}
}
\date{}
\begin{frame}
	\titlepage
\end{frame}

\begin{frame}{Notation, notation notes}
	\begin{description}
		\item[Univariate exposure/treatment] $A$, with realized values $a$ (binary for now)
		\item[Univariate outcome] $Y$, $y$
		\item[Confounders] $\bm{W}$, $\bm{w}$
		\item[Non-confounding covariates] $\bm{V}$, $\bm{v}$
		\item[Individuals] Individuals will be noted with $i$ subscripts, as little as necessary (e.g. $y_{i}$)
		\item[Simplification] Except when necessary, I will introduce most concepts for time-fixed data
	\end{description}
\end{frame}



\begin{frame}[c]{Counterfactuals}
Counterfactuals refer to the \emph{policies} that are counterfactual for each individual: unseen dopplegangers that follow each policy of interest (here ``yes'' or ``no'' exposure)

		\begin{center}
			\begin{tabular}{lcccc}\hline
				id      & W   & A\footnote{Here, I distinguish between "setting" $A$ to some value via a policy versus "observing" $A$ at the same value.}   & Y   &  \\\hline
				1       & 2.9 & yes & no  & factual     \\
				$1_{y}$ & 2.9 & yes & ?   & counterfactual      \\
				$1_{n}$ & 2.9 & no  & ?   & counterfactual      \\
				2       & 0.5 & no  & yes & factual     \\
				$2_y$   & 0.5 & yes & ?   & counterfactual      \\
				$2_n$   & 0.5 & no  & ?   & counterfactual      \\
				\hline
			\end{tabular}
		\end{center}

\end{frame}

%\begin{frame}[c]{Closest possible worlds and counterfactuals}
%	\only<1>{For causal inference, we are often interested in the ``closest possible world'' to our own (we'd like to make inference relevant to our observed world, so we pick a counterfactual that is ``close'' to our own world).}
%	\only<2>{This can be actualized by filling in covariate values such that the counterfactual observations are identical to the observed data in all ways other than possibly the exposure and outcome
%	
%	\begin{center}
%	\begin{tabular}{lcccc}\hline
%id & W & A & Y & counterfactual\\\hline
%1& 2.9& yes & no & no\\
%$1^{yes}$ & \color{red}{2.9}& yes & ?& yes\\
%$1^{no}$& \color{red}{2.9}&  no & ?&  yes\\
%2& 0.5& no & yes& no\\
%$2^{yes}$ & \color{red}{0.5}& yes & ?& yes\\
%$2^{no}$ & \color{red}{0.5}&  no & ?& yes\\
%\hline
%\end{tabular}
%\end{center}
%}
%
%\end{frame}
\begin{frame}[c]{Potential outcomes}
	\only<1>{\textbf{Potential outcome}: $Y^a$, or the value\footnote{These are ``deterministic'' potential outcomes. We could similarly define ``stochastic'' potential outcomes via a probability distribution on $Y^a$.} of the outcome $Y$ we would have observed, had exposure been set to some value $a$}

	\only<2>{An individual (additive) causal effect of ``yes'' vs. ``no'' exposure is defined as $Y_i^{yes}-Y_i^{no}$.\footnote{In the potential outcomes framework, this is a way of saying ``no causation without manipulation'' - causal effects are undefined without hypothetical manipulation.} 
	
	\bigskip
	
	In this sense, \textbf{causal effects are defined without data}, so to learn anything about causal effects we need a way to link what we see (data and/or priors), with what could be (potential outcomes).}

\end{frame}

\begin{frame}[c]{Potential outcomes and causal consistency}
	\only<1>{The link from our factual world to counterfactual observations happens via \textbf{causal consistency}, which is that $Y_i^a = Y_i$ if $A_i=a$. That is, we would expect an individual to have identical outcomes if we set $A=a$ via policy or if we observed $A=a$.\footnote{There is a second assumption here that will be discussed later, which is included in the alternative stable unit treatment value assumption (SUTVA) that was originally used to link counterfactuals with observed data} }
	\only<2>{
		Given causal consistency, we can fill in some of the table
		\begin{center}
			\begin{tabular}{lcccc}\hline
				id        & W                  & A   & Y                &  \\\hline
				1         & 2.9                & yes & no               & factual     \\
				$1^{yes}$ & \color{black}{2.9} & yes & \color{red}{no}  & counterfactual      \\
				$1^{no}$  & \color{black}{2.9} & no  & ?                & counterfactual      \\
				2         & 0.5                & no  & yes              & factual     \\
				$2^{yes}$ & \color{black}{0.5} & yes & ?                & counterfactual      \\
				$2^{no}$  & \color{black}{0.5} & no  & \color{red}{yes} & counterfactual      \\
				\hline
			\end{tabular}
		\end{center}

	}
	\only<3>{
		Here's another way you may see it:\footnote{Rubin, D. B. (1974). Estimating causal effects of treatments in randomized and nonrandomized studies. Journal of Educational Psychology, 66(5), 688–701.}

		\begin{center}
			\begin{tabular}{lccccccc}\hline
				id & W   & A   & Y   & Y$^{yes}$       & Y$^{no}$         \\\hline
				1  & 2.9 & yes & no  & \color{red}{no} & ?                \\
				2  & 0.5 & no  & yes & ?               & \color{red}{yes} \\
				\hline
			\end{tabular}
		\end{center}
	}
\end{frame}


\begin{frame}[t]{The fundamental problem of causal inference}
	\visible<1->{

	Recall that we define an individual, additive\footnote{We could also use ratios and the relative scale.} causal effect as the difference  $Y_i^{yes}-Y_i^{no}$
	\begin{center}
		\begin{tabular}{lcccccccc}\hline
			id & W   & A   & Y   & Y$^{yes}$         & Y$^{no}$  & $Y^{yes}-Y^{no}$          \\\hline
			1  & 2.9 & yes & no  & \color{black}{no} & ?& ?                  \\
			2  & 0.5 & no  & yes & ?                 & \color{black}{yes} & ?\\
			\hline
		\end{tabular}
	\end{center}
	}
	\visible<2>{
		\bigskip

		Even with causal consistency, at least one of the potential outcomes necessary for defining a causal effect will always be missing. This is known as the \textbf{fundamental problem of causal inference}. To make progress we need more assumptions.}
\end{frame}


%\begin{frame}[t]{Aside: The fundamentaler problem of causal inference}
%	\visible<1->{
%		Edwards et al.\footnote{Edwards et al. (2000) \emph{IJE}} demonstrated that, due to measurement error $(W,A,Y)$ are not represented by observed quantities (we see mis-measured versions), so the problem is more accurately represented as
%		\begin{center}
%			\begin{tabular}{lccccccc}\hline
%				id & W & A & Y & Y$^{yes}$ & Y$^{no}$ \\\hline
%				1  & ? & ? & ? & ?         & ?        \\
%				2  & ? & ? & ? & ?         & ?        \\
%				\hline
%			\end{tabular}
%		\end{center}
%	}
%	\visible<2>{
%		\bigskip
%
%		We can't make any progress at all if we don't believe our data (or add assumptions). For today let's believe the data.
%	}
%\end{frame}