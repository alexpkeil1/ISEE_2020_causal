\begin{frame}{Causal inference and mixtures}
	One major barrier to causal inference in mixtures is that none of the simple, textbook examples apply
	
	\bigskip
	
	Today we'll create a map from the things you may know (textbook causality - with review) to things that I hope you can become more comfortable with (causality in mixtures)
	
	\bigskip
	This session will be recorded, and you can access materials, so I will move fast, but please ask questions!
\end{frame}

\begin{frame}{Scope}
	The discussion today is in restricted to a very specific question that might be asked of mixtures data (use your own definition)
	
	\bigskip

       "What would be the health impact on some health outcome if we could modify some or all of the exposures in the mixture?"\footnote{I may speak as though this is the only useful question - forgive my enthusiasm - I do not believe that.}	
	\bigskip

      I prefer to call this endeavor "causal effect estimation"\footnote{Greenland, S. (2017). For and against methodologies: some perspectives on recent causal and statistical inference debates. European journal of epidemiology, 32(1), 3-20.}, but the label "causal inference" persists, so I use it
\end{frame}


\begin{frame}{Causal inference for the extremely busy}
	Causal inference combines (sometimes unverifiable) assumptions and past observations to sharpen knowledge about how we can change the future or what we should have done differently in the past
	
	\bigskip
	
	Causal inference is \textbf{not} the application of special methods that distinguish causation from correlation in a given data set
	\bigskip
	
	Tools of causal inference are effective for evaluating new methods re: do they answer the question I want to ask?
\end{frame}


\begin{frame}{What is a cause?}
	\only<1>{``Cause'' has many potential definitions}
	\only<2>{Many definitions (or at least common uses) are incomplete\footnote{e.g. Granger causality}, ambiguous\footnote{i.e. we cannot map them to precise mathematical statements} or too restrictive\footnote{e.g. deterministic causality}}
%	\only<3>{Consider:\\\includegraphics[width=\linewidth]{fig/acs_cancercauses.png}}
%	\only<4>{Now consider:\\\input{analyses/seer/seercancerdeathrates.tex}\footnotetext{Data from Cancer.gov SEER*Explorer, USA, all race/ethnicities, all sexes}}
%	\only<5>{Why wouldn't we consider age a cause of cancer?}
%	\only<6>{A goal of improving public health leads to a \textbf{policy}\footnote{A ``policy'' is a function that takes as input a current state and outputs an action. This definition comports with common usage of``public health policy'' and so I use in preference to ``regime.''} definition of a cause}
	\only<3>{A goal of improving (and not just observing) public health leads to a \textbf{policy}\footnote{A ``policy'' is a function that takes as input a current state and outputs an action. This definition comports with common usage of ``public health policy'' and so I use in preference to ``regime.''} definition of a cause}
\end{frame}